\chapter*{Introducción} %40 por ciento
\addcontentsline{toc}{chapter}{Introducción}
%%%%%%%%%%%%%%%%%%%%%%%%%%%%%%%%%%%%%%%%%%%%%%%%%%%%%%%%%%%%%%%%
La amplificación de señales débiles es un desafío central en áreas como la física experimental, la ingeniería eléctrica y la detección de radiación. En este contexto, el amplificador de bloqueo (lock-in amplifier) es una herramienta clave para extraer señales ocultas en niveles significativos de ruido, permitiendo obtener mediciones precisas. La importancia de entender el funcionamiento de un amplificador de bloqueo radica en su capacidad para mejorar la relación señal/ruido, algo crucial en aplicaciones donde las señales son extremadamente débiles, como en la detección de partículas subatómicas, la espectroscopía y los sistemas de detección de radiación. Sin embargo, estos dispositivos presentan ciertas limitaciones, como la complejidad de su configuración y su dependencia de la frecuencia de referencia, lo que puede restringir su uso en aplicaciones que requieren respuestas rápidas o frecuencias variables.\\
\\
El objetivo de esta investigación es analizar en detalle el funcionamiento de los amplificadores de bloqueo y su correcta aplicación en la amplificación de señales débiles, explorando tanto sus ventajas como sus limitaciones técnicas. En particular, se busca comprender cómo configurar adecuadamente estos dispositivos para maximizar su efectividad, y qué factores deben considerarse para optimizar su uso en experimentos donde las señales suelen estar sumergidas en ruido. Este tema es especialmente relevante en el campo de la detección de radiación, ya que la precisión en la medición de señales débiles es un factor determinante para obtener resultados fiables.\\
\\
Este trabajo se enfocará en el estudio de un tipo específico de amplificador de bloqueo, profundizando en su funcionamiento, características y aplicaciones prácticas. Aunque existen diversas variantes de estos dispositivos, no se hará una comparación exhaustiva entre ellas, sino que se centrará en su uso básico y su optimización en condiciones experimentales. Con ello, se espera proporcionar un entendimiento claro de cómo estos amplificadores pueden ser una solución eficaz para la amplificación de señales débiles, superando los desafíos asociados a las mediciones de precisión en ambientes ruidosos.
%%%%%%%%%%%%%%%%%%%%%%%%%%%%%%%%%%%%%%%%%%%%%%%%%%%%%%%%%%%%%%%%
\newpage